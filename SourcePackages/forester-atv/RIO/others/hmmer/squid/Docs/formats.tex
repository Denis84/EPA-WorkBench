% --------------------------------------------------------------
% squid:formats.tex
% SRE, Wed Jul 14 17:54:59 1999
% $CVS Id$
% --------------------------------------------------------------

\chapter {Sequence file formats}

\section{Summary}

The software can handle a number of different file formats. By
default, it autodetects the file format, so you don't have to worry
about converting formats. Most common file formats are recognized,
including FASTA, Genbank, EMBL, Swissprot, PIR, and FASTA for
unaligned sequences, and GCG MSF, Clustal, Phylip, and Stockholm
format for multiple sequence alignments. Some parts of the source code
call the autodetector the ``Babelfish''. 

The Babelfish has three drawbacks. First, it takes a small amount of
time to do the autodetection. Second, the Babelfish is aggressive, and
it makes mistakes when a file isn't one of the known formats -- in
particular, it can recognize plain text files as SELEX alignments,
because the SELEX format is so free-form. Third, because the Babelfish
works by reading the first part of the file then rewinding it before
starting to process it, you can't use the Babelfish on a nonrewindable
stream: e.g. when you're taking sequence input from a UNIX pipe
instead of a file, or when the file is gzipped and has to be
decompressed before processing.  In normal use, when you're using the
software interactively from the command line on sequence files that
you're familiar with, the Babelfish is very convenient and
(relatively) safe.

However, you'll find that there are times when you want to override
the Babelfish -- particularly in high-throughput analysis, when you
know the format your files are supposed to be in, and you'd rather
increase robustness and sacrifice interactive flexibility. All the
programs have an \verb+--informat <format>+ option that lets you
specify the format and shut off the Babelfish. You \emph{must} use
\verb+--informat+ to use compressed files, or to read sequence from a
UNIX pipe... see below for more details on these tricks.

\section{Formats recognized by the Babelfish}

Recognized unaligned sequence file formats:

\begin{tabular}{ll}\hline
Format name   & Note \\ \hline
fasta         & BLAST flatfile databases, etc.\\
genbank       & NCBI Genbank flat file format.\\
embl          & Includes both EMBL (DNA) and SWISSPROT (protein) databases.\\
pir           & Protein Information Resource database (NBRF/Georgetown)\\
gcg           & Wisconsin Genetics Computer Group; only allows one sequence per file.\\
gcgdata       & I think this GCG database format is obsolete now.\\ \hline
\end{tabular}

Recognized multiple sequence alignment file formats:

\begin{tabular}{ll}\hline
Format name   & Note \\ \hline
stockholm     & Pfam format. Allows databases of more than one alignment per file\\
selex         & Old NeXagen RNA alignment format, adopted by early HMMER releases.\\
msf           & GCG's alignment format.\\
clustal       & ClustalV, ClustalW, and friends.\\
a2m           & Aligned FASTA format; see comment below.\\
phylip        & Format used by Felsenstein's PHYLIP phylogenetic inference software\\\hline
\end{tabular}

Aligned FASTA format (here called ``A2M'', though I believe that what
Haussler's group at UCSC started calling A2M is yet another variant of
aligned FASTA that's incompatible with this A2M) is only autodetected
when an alignment file is expected. Otherwise an A2M file will be
recognized as unaligned FASTA, and its gap characters (if any) will be
parsed as sequence characters -- often not what you want.

Alignment files may be used when unaligned files are expected -- the
sequences will silently be de-aligned and read sequentially. The
converse is not true; you can't give an unaligned sequence format when
an alignment is expected (makes sense, right?).

There is no provision for enforcing that single unaligned sequence
formats really do contain just a single sequence.  An attempt to
convert a multisequence file to GCG format will silently ``succeed'',
and the file may look ok to your eye, but that multisequence ``GCG''
file is illegal. The data will be corrupted if you try to read that
file back in, possibly without generating any error messages.

It turns out that other formats work too, but they're undocumented,
not subjected to any quality control testing at software release time,
and prone to change without notice at my slightest whim. (In other
words, even less supported than the software already is.) The brave,
curious, or desperate are invited to peruse
\prog{seqio.c} and \prog{squid.h}.

\section{Special tricks}

\subsection{Reading from standard input (probably UNIX-only)}

If you give ``-'' as a sequence filename, the software will read the
sequences from standard input rather than from a file. You will need
to specify the format of the incoming data using the
\verb+--informat+ option.
Any format except SELEX can be read from standard input. This lets you
use any program downstream in a standard UNIX pipe.

There is one limitation: you can't use ``-'' more than once on a
command line, for obvious reasons. (How is it supposed to read more
than one file from one standard input stream?) If you do, behavior of
the software is undefined -- in other words, the software don't check
for whether you're making this mistake, so God help you if you do.

\subsection{Reading from gzip'ed files (probably UNIX-only)}

A sequence file in any format except SELEX can be compressed by gzip,
and read in its compressed form. The software looks for the suffix
\prog{.gz} to detect gzip'ed files. This allows you to save disk space
by keeping sequence files gzip'ed, if you like. gzip is not built in;
the software needs to find a gzip executable in your current PATH.

If for some reason you name a file with a \prog{.gz} suffix and it's
\emph{not} a gzip-compressed file, the software will still try to
decompress it, and peculiar things may happen.

\section{FASTA format, the recommended unaligned format}

FASTA is probably the simplest of formats for unaligned sequences.
FASTA files are easily created in a text editor.  Each sequence is
preceded by a line starting with \verb+>+. The first word on this line
is the name of the sequence. The rest of the line is a description of
the sequence (free format). The remaining lines contain the sequence
itself. You can put as many letters on a sequence line as you want.

\textbf{Example of a simple FASTA file:}
\begin{verbatim}
>seq1 This is the description of my first sequence.
AGTACGTAGTAGCTGCTGCTACGTGCGCTAGCTAGTACGTCA CGACGTAGATGCTAGCTGACTCGATGC
>seq2 This is a description of my second sequence.
CGATCGATCGTACGTCGACTGATCGTAGCTACGTCGTACGTAG CATCGTCAGTTACTGCATGCTCG
\end{verbatim}

For better or worse, FASTA is not a documented standard. Minor (and
major) variants are in widespread use in the bioinformatics community,
all of which are called ``FASTA format''. My software attempts to
cater to all of them, and is tolerant of common deviations in FASTA
format. Certainly anything that is accepted by the database formatting
programs in NCBI BLAST or WU-BLAST (e.g. setdb, pressdb, xdformat)
will also be accepted by my software. Blank lines in a FASTA file are
ignored, and so are spaces or other gap symbols (dashes, underscores,
periods) in a sequence. Other non-amino or non-nucleic acid symbols in
the sequence are also silently ignored, mostly because some people
seem to think that ``*'' or ``.'' should be added to protein sequences
to (redundantly) indicate the end of the sequence. The parser will
also accept unlimited line lengths, which allows it to accomodate the
enormous description lines in the NCBI NR databases.

On the other hand, any FASTA files \emph{generated} by my software
adhere closely to community standards, and should be usable by other
software packages (BLAST, FASTA, etc.) that are more picky about
parsing their input files. That means you can run a sloppy FASTA file
thru \prog{sreformat} to clean it up.

Partly because of this tolerance, the software may have a difficult
time dealing with files that are \textit{not} in FASTA format,
especially if you're relying on the Babelfish to do format
autodetection.  Some (now mercifully uncommon) file formats are so
similar to FASTA format that they be erroneously called FASTA by the
Babelfish and then quietly but lethally misparsed. An example is the
old NBRF file format. If you're using \verb+--informat+, things will
be more robust, and the software should simply refuse to accept a
non-FASTA file -- but you shouldn't count on this, because files
perversely similar to FASTA will still confuse the parser.  (The gist
of these caveats applies to all formats, not just FASTA.)

\section{SELEX, the quick and dirty alignment format}

An example of a simple SELEX alignment file:

\begin{verbatim}
# Example selex file

seq1     ACGACGACGACG.
seq2     ..GGGAAAGG.GA
seq3     UUU..AAAUUU.A

seq1  ..ACG
seq2  AAGGG
seq3  AA...UUU
\end{verbatim}

SELEX is an interleaved multiple alignment format that arose as a
simple, intuitive format that was easy to write and manipulate
manually in a text editor. It is usually easy to convert other
alignment formats into SELEX format, even with a couple of lines of
Perl, but it can be harder to go the other way, since SELEX is more
free-format than other alignment formats. For instance, GCG's MSF
format and the output of the CLUSTALV multiple alignment program are
similar interleaved formats that can be converted to SELEX just by
stripping a small number of non-sequence lines out. Because SELEX
evolved to accomodate different user input styles, it is very tolerant
of various inconsistencies such as different gap symbols, varying line
lengths, etc.

Each line contains a name, followed by the aligned sequence. A space,
dash, underscore, or period denotes a gap. If the alignment is too
long to fit on one line, the alignment is split into multiple blocks,
separated by blank lines. The number of sequences, their order, and
their names must be the same in every block (even if a sequence has no
residues in a given block!) Other blank lines are ignored. You can add
comments to the file on lines starting with a \verb+#+.

SELEX stands for ``Systematic Evolution of Ligands by Exponential
Enrichment'' -- it refers to the Tuerk and Gold technology for
evolving families of small RNAs for particular functions
\cite{Tuerk90b}. SELEX files were what we used to keep track of
alignments of these small RNA families, at a company then called
NeXagen, in Boulder. It's an interesting piece of historical baggage.
With the development of HMMER and more need for annotated alignments
in Pfam, SELEX format later evolved into ``extended SELEX'', with a
reserved comment style that allowed structural markup and other
annotations, but that became unwieldy. We now use Stockholm format
(see below) for highly annotated alignments. (Extended SELEX is
deprecated and undocumented.) Still, the basic SELEX format remains a
useful ``lowest common denominator'' alignment format, and has been
retained.

\subsubsection {Detailed specification of a SELEX file}

\begin{enumerate}
\item
Any line beginning with a \verb+#=+ as the first two characters is a
parsed machine comment in extended SELEX, and is now deprecated. 

\item
All other lines beginning with a \verb+%+ or \verb+#+ as the first
character are user comments.  User comments are ignored by all
software. Anything may appear on these lines. Any number of comments
may be included in a SELEX file, and at any point.

\item
Lines of data consist of a name followed by a sequence. The total
length of the line must be smaller than 4096 characters.

\item
Names must be a single word. Any non-whitespace characters are
accepted.  No spaces are tolerated in names: names MUST be a
single word. Names must be less than 32 characters long.

\item In the sequence, any of the characters \verb+-_.+ or a space are
recognized as gaps. Any other characters are interpreted as sequence.
Sequence is case-sensitive. There is a common assumption by my
software that upper-case symbols are used for consensus (match)
positions and lower-case symbols are used for inserts. This language
of ``match'' versus ``insert'' comes from the hidden Markov model
formalism \cite{Krogh94}. To almost all of my software, this isn't
important, and it immediately converts the sequence to all upper-case
after it's read.

\item
Multiple different sequences are grouped in a block of data lines.
Blocks are separated by blank lines. No blank lines are tolerated
between the sequence lines in a block. Each block in a multi-block
file of a long alignment must have its sequences in the same order in
each block. The names are checked to verify that this is the case; if
not, only a warning is generated. (In manually constructed files, some
users may wish to use shorthand names in subsequent blocks after an
initial block with full names -- but this isn't recommended.)
\end{enumerate}

\section{Stockholm, the recommended multiple sequence alignment format}

While we recommend a community standard format (FASTA) for unaligned
sequence files, the recommended multiple alignment file format is not
a community standard.  The Pfam Consortium developed a format (based
on extended SELEX) called ``Stockholm format''.  The reasons for this
are two-fold. First, there really is no standard accepted format for
multiple sequence alignment files, so we don't feel guilty about
inventing a new one. Second, the formats of popular multiple alignment
software (e.g. CLUSTAL, GCG MSF, PHYLIP) do not support rich
documentation and markup of the alignment.  Stockholm format was
developed to support extensible markup of multiple sequence
alignments, and we use this capability extensively in both RNA work
(with structural markup) and the Pfam database (with extensive use of
both annotation and markup).

\subsection{A minimal Stockholm file}
\begin{verbatim}
# STOCKHOLM 1.0

seq1  ACDEF...GHIKL
seq2  ACDEF...GHIKL
seq3  ...EFMNRGHIKL

seq1  MNPQTVWY
seq2  MNPQTVWY
seq3  MNPQT...

\end{verbatim}

The simplest Stockholm file is pretty intuitive, easily generated in a
text editor. It is usually easy to convert alignment formats into a
``least common denominator'' Stockholm format. For instance, SELEX,
GCG's MSF format, and the output of the CLUSTALV multiple alignment
program are all similar interleaved formats.

The first line in the file must be \verb+# STOCKHOLM 1.x+, where
\verb+x+ is a minor version number for the format specification
(and which currently has no effect on my parsers). This line allows a
parser to instantly identify the file format.

In the alignment, each line contains a name, followed by the aligned
sequence. A dash or period denotes a gap. If the alignment is too long
to fit on one line, the alignment may be split into multiple blocks,
with blocks separated by blank lines. The number of sequences, their
order, and their names must be the same in every block. Within a given
block, each (sub)sequence (and any associated \verb+#=GR+ and
\verb+#=GC+ markup, see below) is of equal length, called the
\textit{block length}. Block lengths may differ from block to block;
the block length must be at least one residue, and there is no
maximum.  

Other blank lines are ignored. You can add comments to the file on
lines starting with a \verb+#+.

All other annotation is added using a tag/value comment style. The
tag/value format is inherently extensible, and readily made
backwards-compatible; unrecognized tags will simply be ignored. Extra
annotation includes consensus and individual RNA or protein secondary
structure, sequence weights, a reference coordinate system for the
columns, and database source information including name, accession
number, and coordinates (for subsequences extracted from a longer
source sequence) See below for details.

\subsection{Syntax of Stockholm markup}

There are four types of Stockholm markup annotation, for per-file,
per-sequence, per-column, and per-residue annotation:

\begin{wideitem}
\item {\emprog{#=GF <tag> <s>}}
	Per-file annotation. \prog{<s>} is a free format text line
	of annotation type \prog{<tag>}. For example, \prog{#=GF DATE
	April 1, 2000}. Can occur anywhere in the file, but usually
	all the \prog{#=GF} markups occur in a header.

\item {\emprog{#=GS <seqname> <tag> <s>}}
	Per-sequence annotation. \prog{<s>} is a free format text line
	of annotation type \prog{tag} associated with the sequence
	named \prog{<seqname>}. For example, \prog{#=GS seq1
	SPECIES_SOURCE Caenorhabditis elegans}. Can occur anywhere
	in the file, but in single-block formats (e.g. the Pfam
	distribution) will typically follow on the line after the
	sequence itself, and in multi-block formats (e.g. HMMER
	output), will typically occur in the header preceding the
	alignment but following the \prog{#=GF} annotation.

\item {\emprog{#=GC <tag> <...s...>}
	Per-column annotation. \prog{<...s...>} is an aligned text line
	of annotation type \prog{<tag>}.
        \verb+#=GC+ lines are
	associated with a sequence alignment block; \prog{<...s...>}
	is aligned to the residues in the alignment block, and has
	the same length as the rest of the block.
	Typically \verb+#=GC+ lines are placed at the end of each block.

\item {\emprog{#=GR <seqname> <tag> <.....s.....>}
	Per-residue annotation. \prog{<...s...>} is an aligned text line
	of annotation type \prog{<tag>}, associated with the sequence
	named \prog{<seqname>}. 
	\verb+#=GR+ lines are 
	associated with one sequence in a sequence alignment block; 
	\prog{<...s...>}
	is aligned to the residues in that sequence, and has
	the same length as the rest of the block.
	Typically
        \verb+#=GR+ lines are placed immediately following the
	aligned	sequence they annotate.
\end{wideitem}

\subsection{Semantics of Stockholm markup}

Any Stockholm parser will accept syntactically correct files, but is
not obligated to do anything with the markup lines. It is up to the
application whether it will attempt to interpret the meaning (the
semantics) of the markup in a useful way. At the two extremes are the
Belvu alignment viewer and the HMMER profile hidden Markov model
software package.

Belvu simply reads Stockholm markup and displays it, without trying to
interpret it at all. The tag types (\prog{#=GF}, etc.) are sufficient
to tell Belvu how to display the markup: whether it is attached to the
whole file, sequences, columns, or residues.

HMMER uses Stockholm markup to pick up a variety of information from
the Pfam multiple alignment database. The Pfam consortium therefore
agrees on additional syntax for certain tag types, so HMMER can parse
some markups for useful information. This additional syntax is imposed
by Pfam, HMMER, and other software of mine, not by Stockholm format
per se. You can think of Stockholm as akin to XML, and what my
software reads as akin to an XML DTD, if you're into that sort of
structured data format lingo.

The Stockholm markup tags that are parsed semantically by my software
are as follows:

\subsubsection{Recognized #=GF annotations}
\begin{wideitem}
\item [\emprog{ID  <s>}] 
	Identifier. \emprog{<s>} is a name for the alignment;
	e.g. ``rrm''. One word. Unique in file.

\item [\emprog{AC  <s>}]
	Accession. \emprog{<s>} is a unique accession number for the
	alignment; e.g. 
	``PF00001''. Used by the Pfam database, for instance. 
	Often a alphabetical prefix indicating the database
	(e.g. ``PF'') followed by a unique numerical accession.
	One word. Unique in file. 
	
\item [\emprog{DE  <s>}]
	Description. \emprog{<s>} is a free format line giving
	a description of the alignment; e.g.
	``RNA recognition motif proteins''. One line. Unique in file.

\item [\emprog{AU  <s>}]
	Author. \emprog{<s>} is a free format line listing the 
	authors responsible for an alignment; e.g. 
	``Bateman A''. One line. Unique in file.

\item [\emprog{GA  <f> <f>}]
	Gathering thresholds. Two real numbers giving HMMER bit score
	per-sequence and per-domain cutoffs used in gathering the
	members of Pfam full alignments. See Pfam and HMMER
	documentation for more detail.
	
\item [\emprog{NC  <f> <f>}]
	Noise cutoffs. Two real numbers giving HMMER bit score
	per-sequence and per-domain cutoffs, set according to the
	highest scores seen for unrelated sequences when gathering
	members of Pfam full alignments. See Pfam and HMMER
	documentation for more detail.

\item [\emprog{TC  <f> <f>}]
	Trusted cutoffs. Two real numbers giving HMMER bit score
	per-sequence and per-domain cutoffs, set according to the
	lowest scores seen for true homologous sequences that
	were above the GA gathering thresholds, when gathering
	members of Pfam full alignments. See Pfam and HMMER
	documentation for more detail.
\end{wideitem}

\subsection{Recognized #=GS annotations}

\begin{wideitem}
\item [\emprog{WT  <f>}]
	Weight. \emprog{<f>} is a positive real number giving the
	relative weight for a sequence, usually used to compensate
	for biased representation by downweighting similar sequences.	
	Usually the weights average 1.0 (e.g. the weights sum to
	the number of sequences in the alignment) but this is not
	required. Either every sequence must have a weight annotated, 
	or none	of them can.  

\item [\emprog{AC  <s>}]
	Accession. \emprog{<s>} is a database accession number for 
	this sequence. (Compare the \prog{#=GF AC} markup, which gives
	an accession for the whole alignment.) One word. 
	
\item [\emprog{DE  <s>}]
	Description. \emprog{<s>} is one line giving a description for
	this sequence. (Compare the \prog{#=GF DE} markup, which gives
	a description for the whole alignment.)
\end{wideitem}


\subsection{Recognized #=GC annotations}

\begin{wideitem}
\item [\emprog{RF}]
	Reference line. Any character is accepted as a markup for a
	column. The intent is to allow labeling the columns with some
	sort of mark.
	
\item [\emprog{SS_cons}]
	Secondary structure consensus. For protein alignments,
	DSSP codes or gaps are accepted as markup: [HGIEBTSCX.-_], where
	H is alpha helix, G is 3/10-helix, I is p-helix, E is extended
	strand, B is a residue in an isolated b-bridge, T is a turn, 
	S is a bend, C is a random coil or loop, and X is unknown
	(for instance, a residue that was not resolved in a crystal
	structure). For RNA alignments
	the symbols \verb+>+ and \verb+<+ are
	used for base pairs (pairs point at each other).  \verb-+- indicate
	definitely single-stranded positions, and any gap symbol indicates
	unassigned bases or single-stranded positions.  This description
	roughly follows \cite{Konings89}. 
	RNA pseudoknots are represented by alphabetic characters, with upper
	case letters representing the 5' side of the helix and lower case
	letters representing the 3' side. Note that this limits the
	annotation to a maximum of 26 pseudoknots per sequence.
	

\item [\emprog{SA_cons}]
	Surface accessibility consensus. 0-9, gap symbols, or X are
	accepted as markup. 0 means <10\% accessible residue surface
	area, 1 means <20\%, 9 means <100\%, etc. X means unknown
	structure.
\end{wideitem}

\subsection{Recognized #=GR annotations}

\begin{wideitem}
\item [\emprog{SS}]
	Secondary structure consensus. See \prog{#=GC SS_cons} above.
\item [\emprog{SA}]
	Surface accessibility consensus. See \prog{#=GC SA_cons} above.
\end{wideitem}


